\documentclass{article}

\usepackage[linkcolor=black,citecolor=red,urlcolor=cyan,colorlinks=true]{hyperref}

\title{\textsc{Report for 1DV516 - PA2\\\large{Algorithms and Advanced Data Structures}}}
\author{\textsc{Ingmar Immanuel Falk}}
\date{\textsc{\today}}

\begin{document}

\maketitle
\pagebreak

\section{Problem 5 - \textit{Isomorphism}}

To check if two trees are isomorphic we have to check if it is possible to obtain
one of the trees by swapping the left and right subtrees of the other tree. This means,
that we have to recursively iterate over every node of \emph{both} trees to confirm
either that the trees are isomorphic or that they are not.

The algorithm consists of only four parts:
\begin{itemize}
    \item Check if both trees are empty. If so, they are isomorphic.
    \item Check if one of the trees is empty. If so, they are not isomorphic.
    \item Check if the values of nodes not are equal. If so, they are not isomorphic.
    \item Recursively check if either both the left and right side of both trees are 
    isomorphic or if the left side of one tree is isomorphic with the right side of the other tree and vice versa.
\end{itemize}

\subsection{Pseudocode}

\begin{verbatim}
    isIso(node1, node2):
        if (node1 is empty and node2 is empty) -> return true
        if (node1 is empty or node2 is empty) -> return false
        if (node1.value != node2.value) -> return false
        return (isIso(node1.left, node2.left) and isIso(node1.right, node2.right)) 
            or (isIso(node1.left, node2.right) and isIso(node1.right, node2.left))
\end{verbatim}

As can be seen in the pseudocode, I used a recursive implementation of the
algorithm. It is obviously also possible to do it with a for loop and that might
be better in terms of performance but this implementation is easier to read and
understand.

\section{Time Complexity}

The time complexity of the algorithm is dependant on the number of nodes
in both trees. Let n be the number of nodes in the first tree and m be the
number of nodes in the second tree. The algorithm makes f our recursively calls
to itself. Each of those calls takes $T(n/2)$ or $T(m/2)$ time, which leads to a
total time complexity of $T(n) = 2T(n/2) + 2T(m/2)$. This can be simplified to
$T(n) = 2T(n/2) + 2T(n/2)$ or $T(n) = 4T(n/2)$, as the worst case is that both
trees have the same number of nodes.
To determine the time complexity of this we can use the Master Theorem,
$T(n) = a^T(n/b)+ f(n)$, where $a = 4, b = 2$ and $f(n) = O(n^k log^p n) = O(1) = n^0$,
since any non recursive operations we do are constant time operations. Plugging
these values into the formula gives us $T(n) = 4T(n/2)+n^0$. This is case 1, where
$log_b a > k$, in our case $log_2 4 = 2 > 0 = k$. This means that the time complexity
is $T(n) = O(n^{log_b a}) = O(n^2)$ according to the Master Theorem.

\section{Problem 6}

For the implementation of the balanced binary search tree I chose the AVL tree.
In order to compare the base implementation of the binary search tree with the
AVL tree I reused the benchmarking utility from the first assignment, although
with some slight modifications. I benchmarked both structures with two tests,
For the first test, each iteration included filling the trees with approximately
$50.000$ random numbers. If we then use this formula: $1.44 * log_2 N - 1.328$, we 
can approximate the depth we expect for the AVL tree. which is a depth of
$1.44 * log_2 50.000 - 1.328 = 21.15$.  After running the first test, which was only
for trees with appr $50.000$ nodes, the average height of the AVL trees were $18.6$,
with an average duration of $0.43$ ms of lookup time for a random number in
the range of $0 - 50.000$. The same test for the base binary search tree without
balancing results in an average height of $36.7$, with an average duration of $676.9$
ms lookup time. This is quite a significant difference in terms of depth and the
base implementation is magnitudes slower than the AVL tree.
For the second test, I reduced the number of nodes to $25.000$ via deletions.
This led to an average height of $18.1$ for the AVL tree and $33.3$ for the unbalanced
tree, while the durations ended up at $0.28$ and $174.0$ respectively. While the
tree height did not significantly change for the AVL tree, its lookup time was
reduced by $55\%$. The unbalanced tree on the other hand, had its height reduced
by 3 levels and its lookup time was reduced to $25\%$ of its original value.

\end{document}