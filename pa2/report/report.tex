
\documentclass{article}

\usepackage[linkcolor=black,citecolor=red,urlcolor=cyan,colorlinks=true]{hyperref}

\title{\textsc{Report for 1DV516 - PA2\\\large{Algorithms and Advanced Data Structures}}}
\author{\textsc{Ingmar Immanuel Falk}}
\date{\textsc{\today}}

\begin{document}

\maketitle
\pagebreak
\tableofcontents
\pagebreak

\section{Introduction}



\section{Problem 5 - \textit{Isomorphism}}

To check if two trees are isomorphic we have to check if it is possible to obtain
one of the trees by swapping the left and right subtrees of the other tree. This means,
that we have to recursively iterate over every node of \emph{both} trees to confirm
either that the trees are isomorphic or that they are not.

The algorithm consists of only four parts:
\begin{itemize}
    \item Check if both trees are empty. If so, they are isomorphic.
    \item Check if one of the trees is empty. If so, they are not isomorphic.
    \item Check if the values of nodes not are equal. If so, they are not isomorphic.
    \item Recursively check if either both the left and right side of both trees are 
    isomorphic or if the left side of one tree is isomorphic with the right side of the other tree and vice versa.
\end{itemize}

\subsection{Pseudocode}

\begin{verbatim}
    isIso(node1, node2):
        if (node1 is empty and node2 is empty) -> return true
        if (node1 is empty or node2 is empty) -> return false
        if (node1.value != node2.value) -> return false
        return (isIso(node1.left, node2.left) and isIso(node1.right, node2.right)) 
            or (isIso(node1.left, node2.right) and isIso(node1.right, node2.left))
\end{verbatim}

\section{Problem 6}



\end{document}